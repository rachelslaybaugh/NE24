The final project for this course consist of several items\+:


\begin{DoxyItemize}
\item an \href{#abstract}{\tt abstract} with a propsal for your project, due March 30, 2015;
\item the \href{#project}{\tt project} itself, due May 1, 2015; and
\item a \href{#presentation}{\tt presentation} about the project, given in class on April 27, 2015. 

 \subsubsection*{Project}
\end{DoxyItemize}

Due 2015/05/01

The final project for this class is to submit a pull request to an open source coding project. The project should be in some way related to computational science. The project can be in any language that you want. It can be hosted on Git\+Hub or Bit\+Bucket (which is easy), or anywhere that you can contribute and provide documentation about your contribution. The pull request needs to tangibly contribute to the project in some way. This can be any of\+:


\begin{DoxyItemize}
\item adding new functionality,
\item adding or improving testing,
\item adding or improving documentation,
\item adding or improving exampels, or
\item something else you feel the project would benefit from.
\end{DoxyItemize}

This should allow you to tailor the project to your personal level of skill, while still pushing yourself to grow.

You may work with a {\itshape partner} or in a {\itshape group} of up to three people. If you choose this route, know that the scope of the project must appropriately increase compared to doing a project by yourself.

Places to look for projects\+:


\begin{DoxyItemize}
\item Open source projects organized by technical area that use Sci\+Py\+: \href{http://www.scipy.org/topical-software.html#topic-guides-organized-by-scientific-field}{\tt here}
\item Py\+N\+E, Python for Nuclear Engineering\+: \href{https://github.com/pyne/pyne}{\tt Git\+Hub page} and \href{http://pyne.io/}{\tt website}
\item If you find another useful list, add it here and submit a pull request to this project page \+:).
\end{DoxyItemize}

Note that most codes will have a list of Issues or Tickets that detail code needs. Many of these are labeled as \char`\"{}novice\char`\"{} or \char`\"{}low-\/hanging fruit\char`\"{}, etc. You can also often contact someone from the development team to see if they have suggestions for good ways to get involved. 

 \subsubsection*{Abstract}

Due 2015/03/30

Submit an abstract on b\+Courses that is one page or less (unless you are in a group, in which case it can be longer) discussing what you will do for your project. This should include\+:


\begin{DoxyEnumerate}
\item The project to which you would like to contribute.
\item An outline of what you intend to do. Include specifically the objective you are trying to accomplish and why that will help the project.
\item Major steps required to execute the project.
\item Deadlines associated with each of these steps.
\item What you need to do to accomplish each step (laying out a path to success).
\item If you are working with a partner or in a group, specify how the work will be divided and who will do what. 

 \subsubsection*{Presentation}
\end{DoxyEnumerate}

Due 2015/04/27

We will do in-\/class presentations on the last day of class. The presentation length will depend upon how many groups we get, but for now plan on five minutes. The talk should include\+:


\begin{DoxyItemize}
\item What project you selected and what it is.
\item What you did.
\item Anything awesome you learned in the process. 
\end{DoxyItemize}