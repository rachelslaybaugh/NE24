\documentclass[11pt]{article}
\usepackage[hmargin=1in,vmargin=1in]{geometry}
\usepackage{amsmath}
\usepackage{amsthm}
\usepackage{graphicx}
\usepackage{placeins}
\begin{document}
\def\bs{\textbackslash}
\setlength\parindent{0pt}
\def\reals{\hbox{\rm I\kern-.18em R}}
\def\complexes{\hbox{\rm C\kern-.43em
\vrule depth 0ex height 1.4ex width .05em\kern.41em}}
\def\field{\hbox{\rm I\kern-.18em F}} %symbol for field
\title{Homework Template}
\setcounter{page}{1}
\begin{flushright}
Matthew Deng \\
NE24 Assignment 7 \\
April 7, 2015
\end{flushright}
\begin{flushleft}
Developer Guide: http://dirac.cnrs-orleans.fr/ScientificPython/ScientificPythonManual/
Specific Module: Under Package Geometry, Scientific.Geometry.Transformation \\


I can see the guide being a good starting point. It's easy to find the information I need to. And the package I'm planning on working on is easy to interpret. What I need to analyze now is the code itself for the functions that Scientific Python is capable of. 



\end{flushleft}
\vspace{1ex}





\end{document}