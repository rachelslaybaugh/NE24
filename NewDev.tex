\documentclass[12pt]{article} % You ALWAYS need a document class

%%%% PREAMBLE %%%%

\usepackage[english]{babel} % This allows you to typeset using a language other than English
\usepackage{blindtext} % This is for filling sections with garbage text
\renewcommand{\familydefault}{\sfdefault} % This changes the font to san-serif (sfdefault)
\usepackage{amsmath} % This includes a bunch of math symbols
\usepackage{graphicx} % This is the main package people use for inserting images
\usepackage[left=1in,right=1in,top=1in,bottom=1in]{geometry} % This sets the margins
\usepackage[colorinlistoftodos]{todonotes} % This allows you to make comments
\usepackage{fancyhdr} % use this package if you want power over header/footer
\usepackage{isotope} % this package makes the spacing for isotopes correct \isotope[235]{U}
\usepackage{subcaption} %used for captions for figures side by side
\usepackage{mathtools} %provides extra options for math
\setlength{\parindent}{0pt} % This removes indentation 
\usepackage{hyperref}

\title{New Developers}

\begin{document}

\section{Setting up GitHub}
Before you start using GitHub or even Git, you have to make it available on your computer. Even if it’s already installed, it’s probably a good idea to update to the latest version. You can either install it as a package or via another installer, or download the source code and compile it yourself.

For download of Git visit \href{http://git-scm.com/downloads} {here}

\subsection{Installing Git}

\begin{itemize}
\item \textbf{MAC} 
There are several ways to install but the easiest is probably to install the Xcode Command Line Tools. On Mavericks (10.9) or above you can do this simply by trying to run git from the Terminal the very first time. If you don’t have it installed already, it will prompt you to install it.
\item \textbf{LINUX}


\end{itemize}



\end{document}

